\documentclass[12pt, aspectratio=169]{beamer}
\beamertemplatenavigationsymbolsempty

\setbeamercolor{normal text}{fg=white,bg=black!80}
\setbeamercolor{title}{fg=yellow,bg=black!80}
\setbeamercolor{frametitle}{fg=yellow,bg=black!80}
\setbeamercolor{itemize item}{fg=white}
\setbeamercolor{itemize subitem}{fg=white}
\setbeamercolor{enumerate item}{fg=white}
\setbeamercolor{block title}{fg=white,bg=black}
\setbeamercolor{section in toc}{fg=white}

\usepackage[utf8]{inputenc}
\usepackage{helvet} % options: mathptmx, helvet, avat, bookman, chancery, charter, culer, mathtime, mathptm, newcent, palatino, pifont and utopia

\hypersetup{colorlinks=true,linkcolor=cyan}
\usepackage[type={CC},modifier={by-sa},version={4.0},]{doclicense}

\usepackage{tikz}
\usetikzlibrary{shapes.geometric, arrows}
\tikzstyle{startstop} = [rectangle, rounded corners, minimum width=3cm, minimum height=1cm,text centered, draw=black, fill=red!30]
\tikzstyle{process} = [rectangle, minimum width=3cm, minimum height=1cm, text centered, draw=black, fill=orange!30]
\tikzstyle{arrow} = [very thick,->,>=latex]
\tikzstyle{every node} = [text=black]

\usepackage{listings}
\lstset{language=[LaTeX]TeX,breaklines=true,basicstyle=\tt\normalsize,keywordstyle=\color{red},commentstyle=\color{gray},morekeywords={subsection,subsubsection,citep,citet,setcitestyle,includepdf,includegraphics}}

\title{Introduction to \LaTeX}
\subtitle{Typesetting your thesis with \LaTeX}
\author{Dr. Gerard Capes}
\institute{Research IT\\University of Manchester}
\date{\today{}}
\logo{\includegraphics[width=0.15\paperwidth]{UniOfManchesterLogo.pdf}}

\begin{document}

\begin{frame}
	\titlepage
\end{frame}

\section{Introduction}

\begin{frame}
	\frametitle{Course Structure}
	\tableofcontents
\end{frame}

\begin{frame}{Disclaimer}{What We \textbf{Won't} Be Covering}
	\begin{itemize}
		\item How to install \LaTeX{} on your laptop\\
		(\LaTeX{} is installed on the training PCs).
		\item Nonetheless, installation receipe:\\ \emph{first} (MikTeX / MacTeX / TeXLive) \emph{then} TeXstudio
		\item For staff managed desktop, TeXLive and TeXstudio are available in the \href{http://softwarecentre.itservices.manchester.ac.uk/}{software centre}:\\
		\emph{first} TeXLive\\
		\emph{then} TeXstudio
		\item Every possible option, setting, package or trick
		\item Image editing or drafting diagrams
	\end{itemize}
\end{frame}

\begin{frame}{Learning objectives}
	\begin{itemize}
		\item Decide if \LaTeX{} is for you
		\item Be able to create an article with sections, cross-references, citations and figures
		\item Know how to find answers to typesetting problems
	\end{itemize}
\end{frame}


\section{What Is \LaTeX?}

\begin{frame} \frametitle{What Is \LaTeX?}
	\begin{itemize}
		\item Software for typesetting.
		\item Totally free and available on many platforms.
		\item Popular in publishing technical and/or large documents.
		\item \textbf{A markup language (content and form are separated).}
	\end{itemize}
\end{frame}

\begin{frame}[fragile] \frametitle{Some terminology}
	\begin{itemize}
		\item \TeX{} is a typesetting system. It is focussed on formatting.
		\item \LaTeX{} is a set of macros built on top of \TeX{} and the idea is to focus on content (e.g. \lstinline|\section{}, \documentclass{}|).\\
		\LaTeX{} $\approx$ \TeX{} + macros.
		\item Mik\TeX, Mac\TeX{} and \TeX{}Live are \TeX{} \textit{distributions} (i.e. \LaTeX{} executables and a collection of packages)
		\item TeXstudio is a \LaTeX\ editor (i.e. a convenient way to prepare your source files, and compile them using \LaTeX{})
		\item Overleaf is an online \LaTeX{} editor
	\end{itemize}
\end{frame}


\section{How \LaTeX{} Works}

\begin{frame} \frametitle{The Software}	
	\begin{center}
		\begin{tikzpicture}[node distance=0.13\paperwidth]
			\filldraw [gray] (3.5,-1.3) rectangle (-2.5,-4.85);
			\node at (2.5,-4.25) {{\TeX}Live};
			\node (start) [startstop] {Markup Text (from your text editor)};
			\node (pro1)  [process, below of=start] 	{\LaTeX};
			\node (pro2)  [process, below of=pro1] 		{\TeX};
			\node (stop)  [startstop, below of=pro2] 	{PDF Output};
			\draw [arrow] (start) -- (pro1);
			\draw [arrow] (pro1) -- (pro2);
			\draw [arrow] (pro2) -- (stop);	
		\end{tikzpicture}
	\end{center}
	\label{pic:software}
\end{frame}

\begin{frame}[fragile]{Input files}{Structure of a \LaTeX{} document}
	\hypertarget{sl:structure}{}
	\begin{itemize}
		\item The input for \LaTeX{} is a plain text file with a \textbf{.tex} extension e.g. mypaper.tex. Create this using a text editor or your \LaTeX{} editor (e.g. TeXstudio).
		\item The file \emph{must} have the following structure:
		\begin{lstlisting}[frame=single,gobble=16]
		\documentclass[options]{class}
		<preamble>
		\begin{document}
		content...
		\end{document}
		\end{lstlisting}
		\item Document classes: article, report, book, letter \ldots
		\item Preamble lists \textbf{packages} for add-on functionality e.g.
		\lstinline|\usepackage{hyperref}|
	\end{itemize}
\end{frame}

\begin{frame}{Exercise 1}
	Create a new \LaTeX{} article, ``Hello world!''
\end{frame}

\begin{frame}[fragile]{Input files}{Some demos of fundamental concepts --- exercises to follow}
	% Demonstrate each of these bullet points by creating a new article file
	\begin{itemize}
		\item Most IDEs have an option to create a new file using a template
		\item All whitespace characters (space, tab) are treated the same by \LaTeX{}; consecutive spaces and tabs are treated as one space, and whitespace at the start of a line is generally ignored.
		\item An empty line creates a new paragraph; consecutive empty lines are treated as one.
		\item Special characters: the following characters have a special meaning in \LaTeX{}: \# \$ \% \^{} \& \_{} \{ \} \~{} \textbackslash
		\item \href{https://wch.github.io/latexsheet/latexsheet-a4.pdf}{Commands} have this format \verb|\command[optional parameters]{parameter}|
		\item Package documentation explains options
		\item Text following a \% symbol is ignored by \LaTeX{}
	\end{itemize}
\end{frame}

\begin{frame}[fragile]{Input files}{Introducing some commands for structural elements}
	\begin{itemize}
		\item lists (\texttt{enumerate, itemize, description})
		\begin{lstlisting}[gobble=16]
		\begin{enumerate}
		    \item first item
		\end{enumerate}
		\end{lstlisting}
		\item \lstinline|\section, \subsection, and \subsubsection|
		\item cross-references e.g. check out equation \ref{eq:integraldemo} \lstinline|\label{sec:intro} \ref{sec:intro}|
		\item footnotes\footnote{Example footnote} are generated like this \lstinline|\footnote{text}|
		\item The \lstinline|hyperref| package enables clickable cross-references and pdf bookmarks
	\end{itemize}
\end{frame}

\begin{frame}[fragile]{Input files}{How to typeset mathematical equations}
	\label{mathslide}
	There are two main options:
	\begin{enumerate}
		\item Inline math mode, using dollar signs: \lstinline|$e=mc^2$| e.g. $e=mc^2$ describes mass-energy equivalence
		\item The equation environment (the code below is typeset as equation \ref{eq:integraldemo})
		\begin{lstlisting}[frame=single,gobble=16]
		\begin{equation}
		    y=\int_0^{\infty} f(x)dx
		    \label{eq:integraldemo}
		\end{equation}
		\end{lstlisting}
		\begin{equation}
			y=\int_0^{\infty} f(x)dx
			\label{eq:integraldemo}
		\end{equation}
	\end{enumerate}
	Some \LaTeX{} editors have a Math menu.\\
	See also Ch. 3 of the \href{https://tobi.oetiker.ch/lshort/lshort.pdf}{not-so-short introduction to \LaTeX{}}
\end{frame}

\begin{frame}{Exercise 2}
	\begin{itemize}
		\item whitespace
		\item paragraphs
		\item commands
		\item lists
		\item inline maths mode
	\end{itemize}
\end{frame}

\begin{frame} \frametitle{Is it For Me?}
	\framesubtitle{A few pros and cons}
	\only<1>{
		Advantages
		\begin{itemize}
			\item Produces beautiful, professional documents
			\item Behaviour is reproducible, so can be debugged c.f. MS Word voodoo
			\item Lots of packages (with documentation) to cater to your typesetting needs e.g. \href{http://www.ctan.org/}{CTAN}
			\item Avoid Word version-compatibility problems
			\item Footnotes, references, bibliographies, mathematical formulae easy to generate
			\item Easier to manage large documents (smaller file size, images and chapters are separate files)
			\item You can version-control your documents
		\end{itemize}
	}
	\only<2>{
		Disadvantages
		\begin{itemize}
			\item You will have to research how to solve some typesetting problems (\href{https://texdoc.org/}{texdoc}, stack exchange)
			\item Learning to use a markup language can distract from the process of writing
			\item Your co-authors may be very resistant to commenting on a pdf instead of using `track changes' in Word
		\end{itemize}
		\vfill
		You're really just deciding which set of problems you'd rather deal with
	}
\end{frame}

\begin{frame}[fragile]{Author-Year citations}{Using the natbib package and Bib\TeX{} program}
	\begin{itemize}
		\item \LaTeX{} itself is only capable of numerical citations, with very limited formatting styles.
		\item Natbib is a popular package capable of both numerical and author-year citations, and many formatting styles are available.
		\item Recipe:
		\begin{itemize}
			\item A \textbf{\href{physics.bib}{.bib}} file containing Bib\TeX{}-formatted references
			\item \lstinline|\usepackage{natbib}| in your preamble
			\item \lstinline|\bibliography{mybibfile}| makes your bibliography
			from the file \textbf{mybibfile.bib}.
			\item Set the style using \lstinline|\bibliographystyle{style}|
			Styles include: agu, dcu, plainnat
		\end{itemize}
		\item Generate in-text citations using \lstinline|\citet{key}| and citations in parentheses using \lstinline|\citep{key}|
			\item Demo (exercises to follow)
	\end{itemize}
\end{frame}

\begin{frame}{Generating a Bib\TeX{} database}
	\begin{itemize}
		\item There are a few example citations in the \textbf{.bib} file for you to practice with
		\item You can generate a Bib\TeX{} database using your reference manager (e.g. \href{https://www.mendeley.com/}{Mendeley})
		\item Many journals have a \textit{download Bib\TeX{} citation} option on their website - you can either import this into your reference manager or paste it into your \textbf{.bib} file directly
	\end{itemize}
\end{frame}

\begin{frame}[fragile]{Author-Year citations}{Bibliography styles and alternatives to natbib}
	\begin{itemize}
		\item Setting the bibliography style automatically sets the citation style to match
		\item Override citation defaults using \lstinline|\setcitestyle{}| in the preamble (see documentation for details)
		\item Many academic journals insist on natbib, and hence a large number of suitable bibliography style files (\textbf{.bst} files) already exist
		\item \textbf{.bst} files are often available from the journal if they're not already present in your installation
		\item Demo from \href{http://www.atmospheric-chemistry-and-physics.net/for_authors/manuscript_preparation.html}{Atmospheric Chemistry and Physics}
		\item If required, you can define your own style using `latex makebst' on the command line
		\item \href{http://tex.stackexchange.com/questions/25701/bibtex-vs-biber-and-biblatex-vs-natbib/25702}{Alternatives to natbib} include the biblatex package, and the biber	program
	\end{itemize}
\end{frame}

\begin{frame}{Output files}{I only wanted a pdf!}
	The software diagram on slide \ref{pic:software} is a simplification. \LaTeX{} outputs a range of files depending on your input file and build profile. These include:
	\begin{itemize}
		\item \textbf{.log} --- contains the transcript from the last \LaTeX{} run. Lists files loaded and packages used.
		\item \textbf{.aux} --- contains information required for cross-referencing.
		\item \textbf{.bbl} --- bibliography info created by Bib\TeX{} -- subsequently inserted by \LaTeX{} into the document.
		\item \textbf{.blg} --- Bib\TeX{} log file.
		\item \textbf{.synctex.gz} --- contains information for jumping between source code and output file in your IDE (demo).
	\end{itemize}
\end{frame}

\begin{frame}{Output files}{Implications for compiling your document}
	\begin{itemize}
		\item If your \textbf{.tex} file contains cross-references and citations, it will need to be processed several times to resolve the dependencies between the intermediate files. The typical sequence is: \href{http://www.tex.ac.uk/FAQ-usebibtex.html}{\LaTeX~Bib\TeX~\LaTeX~\LaTeX}.
		\item If (some of) your citations show up as question marks, it is worth checking the above
		\item Most IDEs can be set up to perform the above sequence at the click of a button (e.g. \textbf{build} in \TeX studio), as well as offering access to commands individually.  \href{https://tex.stackexchange.com/questions/385072/latex-performs-multiple-runs-when-compiling}{Compile repetitions in TeXstudio}
		\item Don't trust your GUI --- check which commands ran
		\item Alternatively, consider using \href{http://mg.readthedocs.io/latexmk.html}{latexmk} \\ \texttt{Tools > Commands > latexmk}
		\item Demo (TeXworks, configure TeXstudio)
	\end{itemize}
\end{frame}

\begin{frame}{Exercise 3}
	\begin{itemize}
		\item the equation environment: refer back to slide \ref{mathslide} for tips
		\item cross references
		\item citations using the natbib package
	\end{itemize}
\end{frame}

\begin{frame}[fragile]{Floats}{Figures}
	\begin{itemize}
		\item A \textit{float} is a container for something that cannot be broken over a page e.g. figures and tables
		\item To create a figure, use the \texttt{figure} environment.
		\item To display images, your \hyperlink{sl:structure}{preamble} must include \lstinline|\usepackage{graphicx}|,  which gives access to the \lstinline|\includegraphics{}| command
		\begin{lstlisting}[frame=single,gobble=16]
		\begin{figure}
		  \centering
		  \includegraphics[]{imagefile}
		  \label{fig:myfigure}
		\end{figure}
		\end{lstlisting}
	\end{itemize}
\end{frame}

\begin{frame}[fragile]{Floats}{Tables}
	\begin{itemize}
		\item Use the table environment -- no extra packages required
		\begin{lstlisting}[frame=single,gobble=16]
		\begin{table}
		  \centering
		  \begin{tabular}{ l|c r }
		    1 & 2 & 3 \\
		    \hline
		    4 & 5 & 6 \\
		    7 & 8 & 9 \\
		  \end{tabular}
		  \label{tab:mytable}
		  \caption{text}
		\end{table}
		\end{lstlisting}
		\item \href{https://en.wikibooks.org/wiki/LaTeX/Tables}{More examples}		
	\end{itemize}
\end{frame}

\begin{frame}[fragile]{Float placement}
	\begin{itemize}
		\item Generally \LaTeX{} does a good job of positioning floats, but you can pass specifiers to exercise more control over the placement of figures
		\begin{lstlisting}[frame=single,gobble=16]
		\begin{figure}[htbp]
		...
		\end{lstlisting}
		\item \href{https://en.wikibooks.org/wiki/LaTeX/Floats,_Figures_and_Captions}{More on figure and placement options}
		\item \href{https://texfaq.org/FAQ-figurehere}{How to put a float exactly where you want it}
		
		\item Demo (exercises next)
		
	\end{itemize}
\end{frame}

\begin{frame}{Exercise 4}
	\begin{itemize}
		\item the figure environment
		\item including graphics files using the graphicx package
		\item cross references
	\end{itemize}
\end{frame}

\section{Getting Help and Helping Yourself}

\begin{frame}{Error messages and warnings}{Errors prevent compilation; warnings are advisory}
	\begin{itemize}
		\item Sooner or later you'll get an error message or warning when compiling your document
		\item The \LaTeX{} wiki book has a \href{https://en.wikibooks.org/wiki/LaTeX/Errors_and_Warnings}{useful guide} covering common messages you are likely to encounter
		\item \href{https://en.wikibooks.org/wiki/LaTeX/Errors_and_Warnings\#Overfull_hbox}{Overful hbox} is particularly common and means \LaTeX{} wasn't able to make your document as pretty as it would like
		\item The fewer things you change between compilations, the easier it will be to locate problems
		\item Back up your work and save changes frequently: consider using \href{http://app.manchester.ac.uk/rgit}{version control}
	\end{itemize}
\end{frame}

\begin{frame} \frametitle{The Minimum Working Example}
	\begin{itemize}
		\item Designed so other users can understand what you are talking about and re-create the problem
		\item No-one is going to troubleshoot hundreds of lines of code
		\item Remember the thesis document (up next) uses a custom stylesheet --- try to provide MWE's in native document classes
		\item \href{https://texfaq.org/FAQ-minxampl}{How to create a MWE}
		\item The process of creating a MWE often results in you solving the problem on your own
	\end{itemize}
\end{frame}

\begin{frame} \frametitle{Resources}
	\begin{itemize}
		\item \href{http://www.github.com/gcapes/latex-course}{This course} (slides and exercises)
		\item \href{http://tex.stackexchange.com}{\LaTeX{} Stack Exchange}
		\item \href{https://wch.github.io/latexsheet/}{\LaTeX{} Cheat Sheet}
		\item \href{https://en.wikibooks.org/wiki/LaTeX}{The \LaTeX{} Wikibook}
		\item \href{https://www.ctan.org/pkg/excel2latex?lang=en}{Excel to \LaTeX{} Converter} and \href{http://www.tablesgenerator.com/}{\LaTeX{} Table Maker}
		\item \href{http://www.tex.ac.uk/faq/}{\TeX{} users group FAQ}
		\item \href{https://tobi.oetiker.ch/lshort/lshort.pdf}{The not so short introduction to \LaTeXe}
		\item \href{http://www.andy-roberts.net/writing/latex}{Getting to grips with \LaTeX} (a series of tutorials)
		\item \href{https://www.latex-tutorial.com/tutorials/}{www.latex-tutorial.com}
	\end{itemize}
\end{frame}

\begin{frame} \frametitle{Concluding remarks on section 1}
	\begin{itemize}
		\item You (probably) don't have to decide now.
		\item It takes time to learn, but you'll spend time typesetting in any program, and it's a useful skill.
		\item Discuss feedback, notes, version control and typesetting with your supervisor as soon as you can.
		\item Please leave some \href{https://docs.google.com/forms/d/e/1FAIpQLSdfpd8QuG9SPAehY5PBJ7AQdbH_eQcDL0UNbS2Oqs6960BTww/viewform?usp=pp_url&entry.1427428485&entry.1759822899&entry.1444288709=Introduction+to+LaTeX&entry.1409009513&entry.160472735&entry.2083518247&entry.92324155}{feedback} on this course (remember this is also the attendance record)
	\end{itemize}
\end{frame}

\section{The University Thesis Document}

\begin{frame} \frametitle{\href{http://documents.manchester.ac.uk/DocuInfo.aspx?DocID=7420}{Presentation of Theses Policy}}	
\framesubtitle{Failure to follow these instructions may result in the faculty rejecting the thesis for examination.}
	\begin{itemize}
		\only<1>{
			\item ``all theses must be submitted ... as a single [PDF] file''
			\item ``each volume of the print copies must include the approved electronically generated cover-page''
			\item ``students may make available ... a redacted version''
			\item ``double or 1.5 spacing in a font type which ensures readability ... [e.g.] 10 point Arial or 12 point Times''
			\item ``single spacing may be used for quotations, footnotes and references''
		}
		\only<2>{			
			\item ``pages may be single or double-sided''
			\item ``bibliographic citations and references must be consistent''
			\item ``to allow for binding the margin at the binding edge of any page must not be less than 40mm; other margins must not be less than 15mm''
			\item The copyright statement must be included
			\item ``page numbering must consist of one single sequence of arabic numerals ... starting with the title page as page number 1. Page numbers must be displayed on all pages \textbf{EXCEPT} the title page''
		}
		\only<3>{			
			\item ``the main text of the theses should ... be left justified''
			\item ``a list contents [is required]''
			\item ``the final word count ... must be inserted at the bottom of the contents page''
			\item ``lists of tables, figures, diagrams, photographs, abbreviations etc ... recommended ... as appropriate ... immediately after the contents page(s)''
			\item ``A short abstract of the contents of the thesis must be inserted into the thesis. The abstract must not be more than one side of A4.''
		}
		\only<4>{
		\item ``The abstract must include'':
			\begin{itemize}
				\item ``Name of the university''
				\item ``Candidate's full name''
				\item ``Degree Title''
				\item ``Thesis Title''
				\item ``Date''
				\item ``Use[ing] a font size of not less than 12 point''
				\item ``Use single spaced typing''
			\end{itemize}
		}
	\end{itemize}
\end{frame}

\begin{frame}{The University Thesis Document Files}
	\begin{block}{Style Sheet}
		Tells \LaTeX{} how to compose the document.
	\end{block}
	\begin{block}{Thesis Itself}
		The ``backbone'' of the document to which you add chapters.
	\end{block}
\end{frame}

\begin{frame}[fragile] \frametitle{How do Files Link?}
\begin{center}
		\begin{tikzpicture}[node distance=0.15\paperwidth]
			\node (pro1)  [process] 	{Thesis File};
			\node (pro2) [process, right of=pro1, xshift=2cm] {Chapter 1};
			\node (pro3) [process, below of=pro2] {Chapter 2};
			\node (pro4) [process, right of=pro3, xshift=2cm] {\includegraphics[width=1cm,keepaspectratio]{example-image-a}};
			\node (pro5) [process, below of=pro1] {Bib\TeX \ Database};
			\draw (pro1) -- (pro2);
			\draw (pro1) -- (pro3);
			\draw (pro3) -- (pro4);
			\draw (pro1) -- (pro5);
		\end{tikzpicture}
	\end{center}
	Thesis.tex is the document you compile.\\
	The chapters are not complete \LaTeX{} documents; during compilation, the contents of each chapter file are inserted into the main thesis.tex file using the \lstinline|\include{file}| command.
\end{frame}

\begin{frame}[fragile]{Alternative format thesis}{A collection of papers}
	\begin{itemize}
		\item If you are submitting an alternative format thesis (including a collection of published papers) this is easily achieved using the \lstinline|\includepdf{}| command
		\item The required package \texttt{pdfpages} is already in the preamble of the template
	\end{itemize}	
\end{frame}

\begin{frame}{Exercises using the thesis template}
	\begin{itemize}
		\item Exercises 5--14
		\begin{itemize}
			\item A chance to practice what we've covered so far
			\item Optionally try a taster of version control
		\end{itemize}
		\item Try typesetting some of your own text
	\end{itemize}

	Please leave some \href{https://docs.google.com/forms/d/e/1FAIpQLSdfpd8QuG9SPAehY5PBJ7AQdbH_eQcDL0UNbS2Oqs6960BTww/viewform?usp=pp_url&entry.1427428485&entry.1759822899&entry.1444288709=Introduction+to+LaTeX&entry.1409009513&entry.160472735&entry.2083518247&entry.92324155}{feedback} on this course (remember this is also the attendance record).

	Make sure you select the right course!
\end{frame}

\begin{frame}
\doclicenseThis
\end{frame}

\end{document}
