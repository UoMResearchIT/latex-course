\documentclass{beamer}
\beamertemplatenavigationsymbolsempty
\usepackage{tabularx}
\usepackage{tikz}
\usepackage{hyperref}
\hypersetup{colorlinks,linkcolor=,}
\usepackage[type={CC},modifier={by-sa},version={4.0},]{doclicense}
\usetikzlibrary{shapes.geometric, arrows}
\tikzstyle{startstop} = [rectangle, rounded corners, minimum width=3cm, minimum height=1cm,text centered, draw=black, fill=red!30]
\tikzstyle{process} = [rectangle, minimum width=3cm, minimum height=1cm, text centered, draw=black, fill=orange!30]
\tikzstyle{arrow} = [thick,->,>=stealth]

\title{Introduction to \LaTeX}
\subtitle{Typesetting your thesis with \LaTeX}
\author{Dr. Gerard Capes}
\institute{Research IT\\University of Manchester}
\date{\today{}}
\logo{\includegraphics[width=0.15\paperwidth]{UniOfManchesterLogo.pdf}}

\begin{document}

\begin{frame}
	\titlepage
\end{frame}

\section{Introduction}

\begin{frame}
	\frametitle{Course Structure}
	\tableofcontents
\end{frame}

\begin{frame} \frametitle{What We \textbf{Won't} Be Covering}
	\begin{itemize}
		\item How to install \LaTeX \ on a Windows/Mac/Linux system\\
		(install via the \href{http://softwarecentre.itservices.manchester.ac.uk/}{software centre} on staff managed desktop, otherwise google for MikTeX, MacTeX or TeXLive)
		\item Every possible option, setting, package or trick
		\item Image editing or drafting diagrams
	\end{itemize}
\end{frame}

\section{What Is \LaTeX?}

\begin{frame} \frametitle{What Is \LaTeX?}
	\begin{itemize}
		\item Software for typesetting.
		\item Totally free and available on many platforms.
		\item Popular in publishing technical and/or large documents.
		\item \textbf{A markup language (content and form are separated).}
	\end{itemize}
\end{frame}

\begin{frame} \frametitle{Is it For Me?}
	\framesubtitle{A few pros and cons}
	\begin{columns}[t]
		\column{0.5\textwidth}
		Advantages
			\begin{itemize}
				\item Produces beautiful, professional documents.
				\item There is almost certainly an elegant solution to your typesetting problem.
				\item You can logically work out how to solve an error.
				\item Encourages you to focus on structure and content - layout is done for you
				\item `Programming' your document is more fun
			\end{itemize}
		\column{0.5\textwidth}
		Disadvantages
			\begin{itemize}
				\item What you see isn't what you get as you write and edit.
				\item You might have to research how to solve the problem.
				\item You will need to learn how to work with markup language.
				\item Your co-authors may be very resistant to commenting on a pdf instead of using `track changes' in Word
			\end{itemize}
	\end{columns}
\end{frame}

\section{How \LaTeX Works}

\begin{frame}[fragile] \frametitle{Some terminology}
	\begin{itemize}
		\item \TeX{} is the typesetting language/program. It is focussed on formatting.
		\item \LaTeX{} is focussed on content, and includes macros (e.g. \verb|\section, \documentclass, \label|) to make using \TeX{} easier.\\
		\LaTeX{} $\approx$ \TeX{} + macros. Further reading \href{https://www.texpad.com/support/latex/advanced/tex-vs-latex}{here} and  \href{http://www.tex.ac.uk/faq/FAQ-texthings.html}{here}.
		\item Mik\TeX, Mac\TeX{} and \TeX{}Live are \TeX{} distributions
	\end{itemize}
\end{frame}

\begin{frame} \frametitle{The Software}	
	\begin{center}
		\begin{tikzpicture}[node distance=0.15\paperwidth]
			\draw [black] (3.5,-1) rectangle (-2,-4.75);
			\node at (2.5,-4.25) {MiK\TeX};
			\node (start) [startstop] {Markup Text (from your text editor)};
			\node (pro1)  [process, below of=start] 	{\LaTeX};
			\node (pro2)  [process, below of=pro1] 		{\TeX};
			\node (stop)  [startstop, below of=pro2] 	{PDF Output};
			\draw [arrow] (start) -- (pro1);
			\draw [arrow] (pro1) -- (pro2);
			\draw [arrow] (pro2) -- (stop);	
		\end{tikzpicture}
	\end{center}
\end{frame}

\begin{frame}[fragile]{Input files}{Some demos of fundamental concepts}
	% Demonstrate each of these bullet points by creating a new article file
	\begin{itemize}
		\item The input for \LaTeX{} is a plain text file with a .tex extension e.g. mypaper.tex. Create this using a text editor or your \LaTeX{} IDE.
		\item All whitespace characters (space, tab) are treated the same by \LaTeX{}; consecutive spaces and tabs are treated as one space, and whitespace at the start of a line is generally ignored.
		\item An empty line creates a new paragraph; consecutive empty lines are treated as one.
		\item Special characters: the following characters have a special meaning in \LaTeX{}: \# \$ \% \^{} \& \_{} \{ \} \~{} \textbackslash
		\item Commands have this format \verb|\command[optional parameter]{parameter}|
		\item Text following a \% symbol is ignored by \LaTeX{}
	\end{itemize}
\end{frame}

\section{The University Thesis Document}

\begin{frame} \frametitle{\href{http://documents.manchester.ac.uk/DocuInfo.aspx?DocID=7420}{Presentation of Theses Policy, June 2014}}	
\framesubtitle{Failure to follow these instructions may result in the faculty rejecting the thesis for examination.}
	\begin{itemize}
		\only<1>{
			\item ``all theses must be submitted ... as a single [PDF] file''
			\item ``each volume of the print copies must include the approved electronically generated cover-page''
			\item ``students may make available ... a redacted version''
			\item ``double or 1.5 spacing in a font type which ensures readability ... [e.g.] 10 point Arial or 12 point Times''
			\item ``single spacing may be used for quotations, footnotes and references''
		}
		\only<2>{			
			\item ``pages may be single or double-sided''
			\item ``bibliographic citations and references must be consistent''
			\item ``to allow for binding the margin at the binding edge of any page must not be less than 40mm; other margins must not be less than 15mm''
			\item The copyright statement must be included
			\item ``page numbering must consist of one single sequence of arabic numerals ... starting with the title page as page number 1. Page numbers must be displayed on all pages \textbf{EXCEPT} the title page''
		}
		\only<3>{			
			\item ``the main text of the theses should ... be left justified''
			\item ``a list contents [is required]''
			\item ``the final word count ... must be inserted at the bottom of the contents page''
			\item ``lists of tables, figures, diagrams, photographs, abbreviations etc ... recommended ... as appropriate ... immediately after the contents page(s)''
			\item ``A short abstract of the contents of the thesis must be inserted into the thesis. The abstract must not be more than one side of A4.''
		}
		\only<4>{
		\item ``The abstract must include'':
			\begin{itemize}
				\item ``Name of the university''
				\item ``Candidate's full name''
				\item ``Degree Title''
				\item ``Thesis Title''
				\item ``Date''
				\item ``Use[ing] a font size of not less than 12 point''
				\item ``Use single spaced typing''
			\end{itemize}
		}
	\end{itemize}
\end{frame}

\begin{frame} \frametitle{The University Thesis Document Files}
	\begin{block}{Style Sheet}
		Tells \LaTeX \ how to compose the document.
	\end{block}
	\begin{block}{Thesis Itself}
		The ``backbone'' of the document to which you add chapters.
	\end{block}
\end{frame}

\begin{frame} \frametitle{How do Files Link?}
\begin{center}
		\begin{tikzpicture}[node distance=0.15\paperwidth]
			\node (pro1)  [process] 	{Thesis File};
			\node (pro2) [process, right of=pro1, xshift=2cm] {Chapter 1};
			\node (pro3) [process, below of=pro2] {Chapter 2};
			\node (pro4) [process, right of=pro3, xshift=2cm] {\includegraphics[width=1cm,keepaspectratio]{example-image-a}};
			\node (pro5) [process, below of=pro1] {Bib\TeX \ Database};
			\draw (pro1) -- (pro2);
			\draw (pro1) -- (pro3);
			\draw (pro3) -- (pro4);
			\draw (pro1) -- (pro5);
		\end{tikzpicture}
	\end{center}
\end{frame}

\section{Getting Help and Helping Yourself}

\begin{frame} \frametitle{The Minimum Worked Example}	
	\begin{itemize}
		\item Designed so other users can understand what you are talking about and re-create the problem.
		\item No-one is going to troubleshoot hundreds of lines of code.
		\item Remember you are using a custom stylesheet --- try to provide MWE's in native document classes.
	\end{itemize}
\end{frame}

\begin{frame} \frametitle{Resources}
	\begin{itemize}
		\item \href{http://www.github.com/tombishop1/thesis-latex-manc-coursematerials}{The Course Resources}
		\item \href{http://tex.stackexchange.com}{\LaTeX \ Stack Exchange}
		\item \href{https://wch.github.io/latexsheet/}{\LaTeX \ Cheat Sheet}
		\item \href{https://en.wikibooks.org/wiki/LaTeX}{The \LaTeX \ Wikibook}
		\item \href{https://www.ctan.org/pkg/excel2latex?lang=en}{Excel to \LaTeX \ Converter} and \href{http://www.tablesgenerator.com/}{\LaTeX \ Table Maker}
		\item \href{latex-users@listserv.manchester.ac.uk}{latex-users@listserv.manchester.ac.uk}
	\end{itemize}
\end{frame}

\begin{frame} \frametitle{Concluding Remarks}
	\begin{itemize}
		\item You (probably) don't have to decide now.
		\item It takes time to learn, but you'll spend time typesetting in any program, and it's a useful skill.
		\item Discuss feedback, notes, version control and typesetting with your supervisor as soon as you can.
	\end{itemize}
\end{frame}

\begin{frame}
\doclicenseThis
\end{frame}

\end{document}
