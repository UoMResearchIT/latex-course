\documentclass[titlepage,colorlinks=true,urlcolor=blue]{article}
\usepackage[T1]{fontenc}
\usepackage[scaled=0.85]{beramono}
\usepackage{textcomp}
\usepackage{xcolor}
\usepackage{hyperref}
\usepackage{listings}
\usepackage{fullpage}
\usepackage[type={CC},modifier={by-sa},version={4.0}]{doclicense}
\lstset{language=[LaTeX]TeX,breaklines=true,basicstyle=\tt\normalsize,keywordstyle=\color{blue},commentstyle=\color{gray}}

\title{Introduction to \LaTeX{} --- Exercises}
\author{Dr Gerard Capes}
\date{\today}

\begin{document}

\maketitle

\section{Opening \TeX studio}
	\begin{enumerate}
		\item Navigate \texttt{Start > All Programs > Programming Languages > TeXstudio}
		\item You may need to configure the application on first run; accept the defaults. This uses the MiKTeX \LaTeX \ distribution that has already been installed.
		\item Navigate the menus to \texttt{Options >  Configure TeXstudio > Build} and check that the default compiler is \texttt{PdfLaTeX}
		\item Ensure that line numbers are visible in the editor window. If they are not, enable them using:
		
		\texttt{Options > Configure TeXstudio\ldots{} > Show advanced options (bottom left checkbox) > Adv. Editor > Appearance > Show line numbers}
	\end{enumerate}

\section{Hello World!}
	\begin{enumerate}
		\item Create a new document.
		\item In the writing pane, \emph{type} the following lines: 		
			\begin{lstlisting}
\documentclass{article}	
\begin{document}				
	Hello World							
\end{document}					
			\end{lstlisting}	
		\item Save the file.
		\item Click on the \texttt{build and view} button.
	\end{enumerate}	

\section{My first latex document}
	\begin{enumerate}
		\item Open \href{first_latex_example.tex}{first\_latex\_example.tex} in \TeX{}studio
		\item Read through the file, taking note of whitespace and special characters -- these are described in the document
		\item Click \texttt{Build \& View} to check your understanding of what you have just read
		\item Try adding another section or subsection, then \texttt{Build \& View} again
		\item Add a table of contents using the \verb|\tableofcontents| command
		\item Change the font size used for the whole document by changing the options given to the \texttt{documentclass} command e.g. \verb|\documentclass[12pt]{article}|
		\item Add a figure using \href{https://en.wikibooks.org/wiki/LaTeX/Floats,_Figures_and_Captions}{this page} as a guide. Use the \texttt{showers.png} file in this repository, or a file of your own. Note that you will need to add the following line to the \href{http://latex.wikia.com/wiki/LaTeX_preamble}{preamble} of your document in order to use the figure environment: \verb|\usepackage{graphicx}|. Experiment with different options such as \verb|\includegraphics[width=0.75\textwidth,angle=270]{showers}|
		\item Add a label to your figure and use it to cross-reference (refer back to) the figure in the text. Use the example on \href{https://en.wikibooks.org/wiki/LaTeX/Labels_and_Cross-referencing}{this page} as a guide.
		\item If you get stuck, talk to the person next to you, ask the instructor, or look at \href{first_latex_solutions.tex}{first\_latex\_solutions.tex}.
	\end{enumerate}

\section{Download the University Thesis Document}
	The following exercises use version control to track changes to your files. The aim is to demonstrate some of the benefits of using version control, and if you are interested in learning more you can apply to our \href{https://app.manchester.ac.uk/rgit}{version control course}. For now, just type the commands, or feel free to ignore the steps which use version control and focus on learning \LaTeX.
	\begin{enumerate}
		\item A zip file containing the current version of the University Thesis Document can be downloaded from the Github repository, \url{https://github.com/gcapes/latex-thesis-template}, using the \texttt{Clone or download} button. Ensure you have the correct branch selected --- this version of the course requires the \texttt{bibtex} branch to be selected.
		\item Find the downloaded zip file in Windows explorer, then \texttt{right click > extract all}. A good choice of location to extract the zip archive to would be your P drive.
		\item Open \textbf{Git Bash} from the Windows start menu. Enter the following commands to configure Git, modified to use your name and email address:

		\begin{lstlisting}[language=bash,showstringspaces=false]
git config --global user.name "Your Name"
git config --global user.email your.name@manchester.ac.uk
		\end{lstlisting}
		\item In \textbf{Git Bash}, navigate to the folder containing the extracted contents of the zip file using the \texttt{cd} command. Change the path in the example command below as required:

		\begin{lstlisting}[language=bash,showstringspaces=false]
cd P:/latex-course
git init
git add .
git commit -m "Initial commit"
		\end{lstlisting}
	\end{enumerate}
The above commands will create a new Git repository, save a copy of the files, and start tracking any changes you make to them.
		
\section{Customise the University Thesis Document}
	For the following exercises, recompile the \texttt{thesis.tex} document after you have completed each set of changes.
		\begin{enumerate}	
			\item Open \texttt{thesis.tex} in \textbf{TeXstudio}. Find and modify the following lines, using your details:
				\begin{lstlisting}
\begin{document}
...
\title{The University Thesis File}
\author{The Author's name}
% Faculty of Life Sciences people should comment the next line out
\school{The Author's school}
\faculty{The Author's faculty}
				\end{lstlisting}
			\item In \textbf{Git Bash} enter the following command:
			\begin{lstlisting}[language=bash,showstringspaces=false]
git commit -am "Add author details"
			\end{lstlisting}
			\item Find and modify the contents of the abstract (the abstract is located between \lstinline|\beforeabstract| and \lstinline|\afterabstract|).
			\item In \textbf{Git Bash} enter the following command:
				\begin{lstlisting}[language=bash,showstringspaces=false]
git commit -am "Add abstract"
				\end{lstlisting}
			\item Add some acknowledgments in the section immediately after \lstinline|\prefacesection{Acknowledgements}|.
			\item In \textbf{Git Bash} enter the following commands. These will save the changes to \texttt{thesis.tex} and display the changes made since the previous commit:
				\begin{lstlisting}[language=bash,showstringspaces=false]
git commit -am "Add acknowledgements"
git diff HEAD~
				\end{lstlisting}
		\end{enumerate}

\section{Adding a Chapter}
	\begin{enumerate}
		\item In \textbf{TeXstudio}, create a new file for your chapter and save it as \texttt{mychapter.tex}
		\item Type \lstinline|\chapter{Methodology}| in the editor window to define it as the start of a chapter.
		\item Try some different writing commands, using the following examples:
			\begin{lstlisting}[upquote=true]
This is going to include some ``quoted text''.				
Some characters are reserved (i.e. they have a special purpose)
but can be accessed by escaping with a backslash, e.g. \%.
To use italics, try \textit{using this command!}.
Math mode is contained within dollars, e.g. $\delta$.
			\end{lstlisting}
		\item Include your new chapter file in the main \texttt{thesis.tex} file after the other included \texttt{*.tex} files (near the end of the file). Start a new line and use the command \lstinline|\include{mychapter}|.
		\item Recompile the \texttt{thesis.tex} file.
		\item In \textbf{Git Bash} enter the following commands to add your new file to the repository:
		\begin{lstlisting}[language=bash,showstringspaces=false]
git add mychapter.tex
git commit -m "Add mychapter"
		\end{lstlisting}
	\end{enumerate}
	
\section{Referencing}
	\begin{enumerate}
		\item Open the thesis document and ensure the \lstinline|\bibliography{}| command points to the bibliography database (the \texttt{library.bib} file). The \texttt{.bib} file extension is not needed.
		\item Open your new chapter file, start a new paragraph (insert a blank line) and include some referencing commands in the text using references from the bibliography file. Some examples are given below:
		\begin{lstlisting}
\citep{Payne2011}		
\citealp{Porinchu2003,Payne2011}	
\citet{Payne2011}			
\citealt{Payne2011}			
\citep[e.g.][]{Porinchu2003}		
\citep[][p.245]{Porinchu2003}	
		\end{lstlisting}
		\item Commit your changes using the following Git command:
		\begin{lstlisting}[language=bash,showstringspaces=false]
git commit -am "Add paragraph with references"
		\end{lstlisting}
		\item Deliberately introduce a compilation error by inserting a dollar sign in the middle of a command and trying to rebuild the \texttt{thesis.tex} document e.g. break one of the reference commands in your new chapter:
		\begin{lstlisting}
\cit$ep{Payne2011}
		\end{lstlisting}
		Now imagine you hadn't just deliberately introduced an error which is easy to find and fix. There are various approaches to solve this problem. Here is a version control solution to view changes made since the last known working version (your last commit), and then put the modified files back to the last known working version:
		\begin{lstlisting}[language=bash,showstringspaces=false]
git diff HEAD		# Displays changes
git checkout HEAD .	# Restores your previous working state
		\end{lstlisting}
	\end{enumerate}

\section{Cross-referencing}
	\begin{enumerate}
		\item Add a reference label to the new chapter, using the command \lstinline|\label{ch:mychapter}|.
		\item Cross reference the chapter in the \texttt{thesis.tex} file, e.g.
			\begin{lstlisting}
As described in chapter \ref{ch:mychapter}, this document is written in \LaTeX.
			\end{lstlisting}
		\item Remember you'll need to use a unique name inside the curly braces for each section, chapter, figure etc.
		\item Commit your changes
		\begin{lstlisting}[language=bash,showstringspaces=false]
git commit -am "Add cross-reference to mychapter"
		\end{lstlisting}
	\end{enumerate}
	
\section{Adding Figures}
	\begin{enumerate}
		\item Use the following code to add a figure in \texttt{mychapter.tex}:
			\begin{lstlisting}
\begin{figure}[htbp]							
	\centering								
		\includegraphics[width=1\textwidth]{example-image-a}
	\caption[Short caption]{Long caption}			
	\label{labeltexthere}						
\end{figure}
		\end{lstlisting}
		\item Try modifying the terms in the square brackets after \lstinline!\includegraphics!; experiment with \lstinline!\pageheight! or \lstinline!keepaspectratio!. Try using multiple terms separated with a comma (\lstinline!,!). 
		\item Now try the \lstinline!sidewaysfigure! environment. This uses the \texttt{rotating} package, so check whether you already have it in the preamble!
		\item Run the following \textbf{Git} commands to commit your changes, and view a history of your project
			\begin{lstlisting}[language=bash,showstringspaces=false]
git commit -am "Add figure to mychapter"
git log
			\end{lstlisting}
		With version control it's possible to restore your directory or individual files to the same state as any of the commits shown in your history.
	\end{enumerate}
	
\section{Drawing Tables}
	\begin{enumerate}
		\item Create a table using the following lines of code:
			\begin{lstlisting}
\begin{table}[htbp]						
	\centering							
	\begin{tabular}{lcc}					
		\toprule						
			Name    & Age & Gender \\ \midrule	
			Tom     & 26  & Male   \\			
			James   & 22  & Male   \\			
			Eleanor & 25  & Female \\			
		\bottomrule					
	\end{tabular}						
	\caption[Short caption]{Long caption}		
	\label{}							
\end{table}
			\end{lstlisting}
		\item Try adding a column for attendance by adding \lstinline!c! to the \lstinline!\begin{tabular}{lcc}! and some data to each row.
		\item Remember cells are separated with an ampersand (\lstinline!&!), so you'll need to add this to each row.
		\item Consider commiting your changes to your \textbf{Git} repository (and for the remaining exercises).
	\end{enumerate}
	
\section{Writing Equations}
	\begin{enumerate}
		\item Create an equation using the following code as an example:
			\begin{lstlisting}
\begin{equation}			
	e=mc^{2}			
\label{massenergyequiv}	
\end{equation}
			\end{lstlisting}
		\item Reference the equation in the text using \lstinline!\ref{massenergyequiv}!.
	\end{enumerate}

\section{Inserting a quote}
	Try inserting a quote using the \lstinline!\quote! environment:
		\begin{lstlisting}
\begin{quote}
I think therefore I use \LaTeX for typesetting. 
\end{quote}
		\end{lstlisting}

\section{Escaped \& Special Characters}
	Try using reserved characters like \&, \textbackslash, and \%, and you'll run into difficulty. Try writing the following:
		\begin{lstlisting}
If you need to write an ampersand (\&), use the slash to escape
or it will not render.
Similarly, if you wanted to use a slash or backslash (\backslash),
you must use a command.
If you need accents, they are often produced l\'ike this.
		\end{lstlisting}

\begin{figure}[b]
\doclicenseThis
\end{figure}

\end{document}
