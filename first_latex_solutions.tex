% SMALL.TEX -- Released 5 July 1985
% USE THIS FILE AS A MODEL FOR MAKING YOUR OWN LaTeX INPUT FILE.
% EVERYTHING TO THE RIGHT OF A  %  IS A REMARK TO YOU AND IS IGNORED
% BY LaTeX.
%
% WARNING!  DO NOT TYPE ANY OF THE FOLLOWING 10 CHARACTERS EXCEPT AS
% DIRECTED:        &   $   #   %   _   {   }   ^   ~   \

\documentclass[12pt]{article}  % YOUR INPUT FILE MUST CONTAIN THESE
\usepackage{graphicx}	% Required for figure environment
\usepackage{natbib}

\begin{document}                  % TWO LINES PLUS THE \end COMMAND AT
                                  % THE END
\tableofcontents

\section{Simple Text}          % THIS COMMAND MAKES A SECTION TITLE.

Words are separated by one or    more      spaces.  Paragraphs are
    separated by one or more blank lines.  The output is not affected
by adding extra spaces or extra blank lines to the input file.


Double quotes are typed like this: ``quoted text''.
Single quotes are typed like this: `single-quoted text'.

Long dashes are typed as three dash characters---like this.

Italic text is typed like this: \textit{This is italic text}.
Bold   text is typed like this: \textbf{this is  bold  text}.

\subsection{A Warning or Two}        % THIS MAKES A SUBSECTION TITLE.

If you get too much space after a mid-sentence period---abbreviations
like etc.\ are the common culprits)---then type a backslash followed by
a space after the period, as in this sentence.

Remember, don't type the 10 special characters (such as dollar sign and
backslash) except as directed!  The following seven are printed by
typing a backslash in front of them:  \$  \&  \#  \%  \_  \{  and  \}.
The manual tells how to make other symbols.

\section{My own section}
This is another section, added for the purpose of the exercise.
Note that this line doesn't create a new paragraph.

This does though.	I can leave any  combination of spaces and tabs
between words and they're just treated as one.

There are various environments for listing things in \LaTeX{}:
\begin{itemize}
	\item itemize (creates a list of bullets like this one)
	\item enumerate (creates a numbered list)
	\item description (creates a list using descriptions instead of bullets or numbers)
\end{itemize}

\section{Science!}
The relation $e=mc^2$ describes the equivalence between mass and energy \citep{massenergy}.


\section{Weather}
% Include an image file in a figure
\begin{figure}[h]
	\centering
	\includegraphics[width=0.5\textwidth]{showers}
	\caption{typical weather}
	\label{fig:weather}
\end{figure}

Typical British weather is depicted in figure \ref{fig:weather}.

\bibliography{physics}
\bibliographystyle{agu}

\end{document}                    % THE INPUT FILE ENDS LIKE THIS
